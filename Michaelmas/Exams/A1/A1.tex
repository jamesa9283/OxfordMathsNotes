\documentclass[a4paper,10pt]{article}
\usepackage[top=2.5cm,bottom=2.5cm,left=2.5cm,right=2.5cm,showframe]{geometry}
\usepackage{xcolor,fancyhdr}
\usepackage{tikz}
\usepackage{amsmath,amssymb,amsthm,amsfonts,bm}
\usepackage{stmaryrd}
\setlength\parindent{0pt}
\setlength\headheight{14.01pt}
\usepackage{lineno}
\linenumbers
\setpagewiselinenumbers

\newcommand{\inn}[1]{\langle #1 \rangle}

\definecolor{purple}{RGB}{180, 2, 212}
\definecolor{themecolor}{HTML}{00A8FF} % light blue
\definecolor{themecolor2}{HTML}{659157} % olive green
\definecolor{themecolor3}{HTML}{3D518C} % faint blue

\newcommand{\tcb}{\textcolor{blue}}
\newcommand{\tcr}{\textcolor{purple}}
\newcommand{\co}{C_0^{\infty}(\mathbb{R})}

\renewcommand{\vec}[1]{\bm{#1}}
\newcommand{\dx}{\,\mathrm{d}x}
\newcommand{\dy}{\,\mathrm{d}y}
\newcommand{\dz}{\,\mathrm{d}z}
\newcommand{\du}{\,\mathrm{d}u}
\newcommand{\dr}{\,\mathrm{d}r}
\newcommand{\ds}{\,\mathrm{d}s}
\newcommand{\dt}{\,\mathrm{d}t}
\newcommand{\dk}{\,\mathrm{d}k}
\renewcommand{\d}{\,\mathrm{d}}

%%%%%%%%  DO NOT EDIT BELOW %%%%%%%%

\lhead{
  \begin{tikzpicture}
    \hspace{-1.4cm}\filldraw[black] (0,0) circle (5pt) node[anchor=west]{~};
  \end{tikzpicture}
  \hspace{-7mm} Candidate Number: 1069110}
\pagestyle{fancyplain}

\definecolor{notepadrule}{RGB}{240,240,240}

% to show the page frame

\makeatletter
\renewcommand*{\Gm@vrules@mpi}{%
  \hb@xt@\@tempdima{\llap{\Gm@vrule}\ignorespaces
    \hskip \textwidth\Gm@vrule%\hskip \marginparsep
    % \llap{\Gm@vrule}%
    \hfil
    % \Gm@vrule%
  }}%
\renewcommand*{\Gm@vrules@mpii}{%
  \hb@xt@\@tempdima{\hskip-\marginparwidth\hskip-\marginparsep
    % \llap{\Gm@vrule}%
    \ignorespaces
    \hskip \marginparwidth
    % \rlap{\Gm@vrule}%
    \hskip \marginparsep
    \llap{\Gm@vrule}\hskip\textwidth\rlap{\Gm@vrule}\hss}}%
\renewcommand*{\Gm@pageframes}{%
  \vb@xt@\z@{%
    \ifGm@showcrop
      \vb@xt@\z@{\vskip-1\Gm@truedimen in\vskip\Gm@layoutvoffset%
        \hb@xt@\z@{\hskip-1\Gm@truedimen in\hskip\Gm@layouthoffset%
          \vb@xt@\Gm@layoutheight{%
            \let\protect\relax
            \hb@xt@\Gm@layoutwidth{\Gm@cropmark(-1,1,-3,3)\hfil\Gm@cropmark(1,1,3,3)}%
            \vfil
            \hb@xt@\Gm@layoutwidth{\Gm@cropmark(-1,-1,-3,-3)\hfil\Gm@cropmark(1,-1,3,-3)}}%
          \hss}%
        \vss}%
    \fi%
    \ifGm@showframe
      \if@twoside
        \ifodd\count\z@
          \let\@themargin\oddsidemargin
        \else
          \let\@themargin\evensidemargin
        \fi
      \fi
      \moveright\@themargin%
      \vb@xt@\z@{%
        \vskip\topmargin%\vb@xt@\z@{\vss\Gm@hrule}%
        \vskip\headheight%\vb@xt@\z@{\vss\Gm@hruled}%
        \vskip\headsep\vb@xt@\z@{\vss\Gm@hrule}%
        \@tempdima\textwidth
        \advance\@tempdima by \marginparsep
        \advance\@tempdima by \marginparwidth
        \if@mparswitch
          \ifodd\count\z@
            \Gm@vrules@mpi
          \else
            \Gm@vrules@mpii
          \fi
        \else
          \Gm@vrules@mpi
        \fi
        \vb@xt@\z@{\vss\Gm@hrule}%
        \vskip\footskip%\vb@xt@\z@{\vss\Gm@hruled}%
        \vss}%
    \fi%
  }}%
\makeatother

% Numbers (a bit more difficult)
\newcommand{\pageofnumberedlines}[1]{%
  \begingroup\offinterlineskip
  \count255=\vsize \dimen0=#1
  \divide\count255 by \dimen0
  \mathchardef\howmany=\count255
  \count255=0
  \loop\ifnum\count255<\howmany
  \advance\count255 by 1
  \hbox to\hsize{%
    \vrule height#1 width 0pt
    \llap{\scriptsize\number\count255\space\space}%
    { \color{notepadrule}     \hrulefill}
  }
  \repeat
  \endgroup
}

%%%%%%%% %%%%%%%% %%%%%%%% %%%%%%%% %%%%%%%% %%%%%%%%


%%%%%%%%%%%%%  PLEASE DO NOT EDIT ANY OF THE LINES ABOVE %%%%%%%%%%%%%%%
% Insert your text between "\begin{document}" and "\end{document}" below.
% The total length of your summary notes should not exceed 2 sides of a
% single sheet of A4, with maximum 58 lines of text per page.
%%%%%%%%%%%%%%%%%%%%%%%%%%%%%%%%%%%%%%%%%%%%%%%%%%%%%%%%%%%%%%%%%%%%%%%%

\newcommand{\abs}[1]{\left\lvert #1\right\rvert}
\newcommand{\bket}[1]{\left\lvert #1\right\rangle}
\newcommand{\brak}[1]{\left\langle #1 \right\rvert}
\newcommand{\braket}[2]{\left\langle #1\middle\vert #2 \right\rangle}
\newcommand{\bra}{\langle}
\newcommand{\ket}{\rangle}
\newcommand{\norm}[1]{\left\lVert #1\right\rVert}
\newcommand{\normalorder}[1]{\mathop{:}\nolimits\!#1\!\mathop{:}\nolimits}
\newcommand{\tv}[1]{|#1|}
\renewcommand{\vec}[1]{\boldsymbol{\mathbf{#1}}}
\newcommand{\ip}[2]{\left\langle #1\,, #2 \right\rangle}


\newcommand{\fa}{\forall\,}
\newcommand{\ex}{\exists\,}
\renewcommand{\l}{\lambda}

% LA

\newcommand{\fs}{\sum_{i=1}^n}
\newcommand{\vV}{\vec v \in V}
\renewcommand{\v}{\vec v}

% DSaC
\newcommand{\di}[2]{\frac{d #1}{d #2}}
\newcommand{\dit}{\frac{d}{dt}}
\newcommand{\dii}[2]{\frac{d^2 #1}{d #2 ^2}}
\newcommand{\din}[3]{\frac{d^{#1} #2}{d #3 ^{#1}}}
\renewcommand{\d}{\delta}
\renewcommand{\L}{\mathcal{L}}
\newcommand{\e}{\varepsilon}
\renewcommand{\o}{\omega}


% MathsBio
\renewcommand{\a}{\alpha}
\renewcommand{\b}{\beta}
\newcommand{\g}{\gamma}


\newcommand{\D}{\Delta}

% GRF
\newcommand{\s}{\sigma}
\newcommand{\gen}[1]{\left\langle #1 \right\rangle}
\renewcommand{\t}{\tau}
\newcommand{\wt}[1]{\widetilde #1}
\renewcommand{\r}{\rho}
\newcommand{\cS}{\mathcal{S}}
\newcommand{\sub}{\subseteq}
\newcommand{\QX}{\Q[X]}
\newcommand{\sm}{\setminus}
\renewcommand{\bar}[1]{\overline{#1}}

% NT
\newcommand{\m}{\mid}
\newcommand{\nm}{\nmid}
\renewcommand{\c}{\equiv}
\newcommand{\ti}{\times}
\newcommand{\ls}[2]{\left(\frac{#1}{#2}\right)}
\newcommand{\gls}{\ls a p}
\newcommand{\ceil}[1]{\left \lceil #1 \right \rceil }
\newcommand{\floor}[1]{\left \lfloor #1 \right \rfloor }

% Diss
\DeclareMathOperator{\Ad}{Ad}
\newcommand{\mf}[1]{\mathfrak #1}
\renewcommand{\O}{\Omega}
\newcommand{\La}{\Lambda}
\newcommand{\Oh}{\hat{\vec \Omega}}
\newcommand{\Ov}{\vec \Omega}
\newcommand{\Od}{\dot \Ov}
\newcommand{\ov}{\vec \o}
\newcommand{\oh}{\hat{\vec \omega}}
\newcommand{\Lh}{\hat{\vec \Lambda}}
\newcommand{\Lv}{\vec \Lambda}
\newcommand{\Ld}{\dot \Lv}
\newcommand{\Gh}{\hat{\vec \Gamma}}
\newcommand{\Gv}{\vec \Gamma}
\newcommand{\Gd}{\dot \Gv}
\newcommand{\Ga}{\Gamma}
\newcommand{\Lhd}{\boldsymbol{\dot{\hat{\La}}}}
\newcommand{\ohd}{\boldsymbol{\dot{\hat{\o}}}}
\newcommand{\I}{\mathbb{I}}
\newcommand{\J}{\mathbb{J}}
\newcommand{\ditat}[1]{\left.\dit\right|_{t = #1}}
% \renewcommand{\hat}[1]{\widehat{#1}}

% PDEs

\newcommand{\nab}{\mathbf{\nabla}}
\newcommand{\grad}{{\nab}\, f}
\renewcommand{\div}{\nab \cdot}
\newcommand{\curl}{\nab \times}
\newcommand{\cc}{\mathcal{C}}
\newcommand{\pdxy}[1]{\frac{\partial #1}{\partial x\partial y}}
\newcommand{\Th}{\Theta}
\newcommand{\cF}{\mathcal{F}}

% TMs
\newcommand{\T}{\mathcal{T}}
\newcommand{\cI}{\mathcal{I}}
\newcommand{\vn}{\varnothing}
\newcommand{\cB}{\mathcal{B}}
\newcommand{\G}{\Gamma}


\let\Im\relax
\let\Re\relax

% MAGIC
%% AlgTop
\newcommand{\fg}[1]{\pi_1( #1 )}
\DeclareMathOperator{\map}{Map}


% NumLinAlg
\renewcommand{\S}{\Sigma}

%DiffMan
\newcommand{\ot}{\otimes}

\newcommand{\pd}[2]{\frac{\partial #1}{\partial #2}}
\newcommand{\pdd}[2]{\frac{\partial^2 #1}{\partial #2 ^2}}


\begin{document}
\textbf{APDEs}
\textit{Just go for the $\t$ method, it will make your life better. Trust me.}\\
\noindent
Uniqueness. $\mathcal{J} = 0$, solve, plug into $x(t, s)$. chars ($x(t)$) at end pts $s$, for bounds, sketch the chars and envelope. Chars are tangent to envelope. \\
\noindent
\textbf{Grns Thm}: $\int_\Gamma (P\mathrm{d}y - Q\mathrm{d}x) = \iint_\Gamma P \frac{\partial }{\partial x}(P\psi) + \frac{\partial }{\partial y}(Q\psi) $\\
\noindent
Shock is causality when $\left[ \frac{\mathrm{d}x}{dt} \right]_- > \frac{dX}{dt} > \left[ \frac{\mathrm{d}x}{dt} \right]_+$ and $\frac{dX}{dt} = \frac{[Q]_-^+}{[P]_-^+}$\\
\noindent
\textbf{Charpits} $\di x \t = \pd F p, \quad \di y \t  = \pd F q, \quad \di p \t = - \pd F x - p \pd F u,  \quad \di q \t = -\pd F y - q\pd F u, \quad \di u \t = p\pd F p + q\pd F q.$\\
For BCs, $u_0' = p_0x_0' + q_0y_0'$ and use PDE as second eqn.\\
\noindent
Optics, $\psi (x, y, t) = \phi(x, y)e^{-i\o t}$. WKBJ, $\phi(x, y) = A(x, y)e^{iku(x, y)}$.\\
\textbf{Systems} $\mathbf A(x, y, \mathbf u) \pd {\mathbf u} x + \mathbf B(x, y, \mathbf u)\pd {\mathbf u} y = \mathbf c(x, y, \mathbf u)$. Note $\di x t = \l$ where $\det(\mathbf B - \l \mathbf A) = 0$. For $2 \ti 2$, if 2 $\l$ then hyperbolic, if $\l_1 = \l_2$ then parabolic if just 1 $\l$ then elliptic.\\
\noindent
{Solve}, find left evec $\mathbf \ell$ of $\mathbf B - \mathbf\l \mathbf A$, then $\mathbf\ell^T \mathbf A \di {\mathbf u} x = \mathbf\ell^T\mathbf c$. Try to intergrate the solns to produce Riemann invariants, e.g.
$ \di{}{x} (u \mp v) = u + v$ on $\di{y}{x} = \pm 1$ implies that $e^x (u \mp v) = c$ on $\di y x = \pm 1$ and so we have $e^x(u \mp v) = f(y \mp x)$. Now intergate and solve for $u$ and $v$.\\
Region of influence is where we solve $u$ and $v$ for the functions, plug in initial data and then replug those into $u$ and $v$. Then the region is union of where funcs are defined.\\
\noindent
Rankine–Hugoniot condition for systems of PDEs is, $(B - \pd u x A )[u]_-^+ = 0$ so need $\det (B - \pd u x A) = 0$\\
\noindent
\textbf{Second Order PDEs:} $au_{xx} + 2b u_{xy} + cu_{yy} = f$. The Cauchy data we need for this is, $x = x_0(s)$, $y = y_0(s)$, $u = u_0(s)$ and $\pd u n = v_0(s)$. However,
\begin{center}
  $\displaystyle{u_0'(s) = \pd u x x_0'(s) + \pd u yy_0'(s)}$
\end{center}
\begin{center}
  $\displaystyle{ v_0'(s) = \frac{\pd u x x_0'(s) + \pd u yy_0'(s)}{\sqrt{x_0'^2 + y_0'^2}}}$
\end{center}
So replace $\pd u n = v_0(s)$ with $\pd u x = p_0(s)$ and $\pd u y = q_0(s)$. Then diff $p_0$ and $q_0$ to get the uniqueness cond. Further note $a\l^2 \mathbf{-} 2b\l + c = 0$. \textbf{MINUS}.\\
\noindent
Hyperbolic: $u_{\xi\eta} = \phi(\xi, \eta, u, u_\xi, u_\eta)$. Parabolic: $u_{\xi\xi} = \phi$. Elliptic, $\xi \pm i\eta = K$, then $u_{\xi\xi} + u_{\eta\eta} = \phi$.\\

\noindent
Number of BCs needed on boundary equal to number of chars travelling out of boundary. Two chars out, boundary is space-like, two BCs. time-like boundary, one char in and one out, one condition.\\
\noindent
\textbf{Riemann Functions:} $\mathcal{L}[u] = u_{xy} + au_x + bu_y + cu = f$ and adjoint is bring coeffs in and make odd derivs minus.
\begin{center}
  $\displaystyle{v \mathcal{L}[u] - u \mathcal{L}^*[v]} = \pd{}{x} \left(v \pd u y + auv\right) + \pd{}{y} \left( -u\pd v x + buv \right)$
\end{center}
Now take the double integral over $D$, $v = R$, , then Grns Thm, impose $\mathcal{L}^*[R] = 0$ and then $\mathrm{d}y = 0$ on AP and $\mathrm{d}x = 0$ on BP. Then $R_x = bR$, $R_y = aR$ and $R = 1$ at P. Then we have the properties,\\
\noindent
$\mathcal{L}^*[R] = 0$, $R_x = bR$ ($y= \eta$), $R_y = aR$ ($x = \xi$) and $R = 1$ on $(x, y) = (\xi, \eta)$.\\
\noindent
\textbf{Uniqueness, Poisson.} Take $u = u_1 - u_2$. Then consider,
\begin{center}
  $\displaystyle{\iint_D \nab \cdot (\phi \nab \phi)\mathrm{d}x\mathrm{d}y = \iint_D \pd{}{x}\left(\phi \pd \phi x\right) + \pd{}{y}\left(\phi \pd \phi y\right)\mathrm{d}x\mathrm{d}y}$
\end{center}
Then greens leads to $|\nab \phi| = 0$\\
\noindent
Let $\xi = \e^a x$, $\t = \e^b t$ and $w = \e^c u(\e^a x, \e^b t)$. Then rewrite in terms of these variables and simplify. Then set the power of $\e$ be zero. Now, $\e^a = \xi / x$, $\e^b = \t / t$. Hence $(\t / t)^{a/b} = \xi / x$ and $\xi / \t^{a/b} = x / t^{a/b} = \eta$ and $w / \t^{c/b} = u/t^{c/b} = \nu$. Then $u = f(\eta)$ and hey presto!

\newpage
\noindent
Inhomogenous Systems ($\mathcal{L}[u] = f(x)$ subject to $\mathcal{B}[u] = g(x)$), solve $\mathcal{L}[u_1] = f(x)$ with $\mathcal{B}[u_1] = 0$ and then $\mathcal{L}[u_2] = 0$ with $\mathcal{B}[u_2] = g(x)$ and then $u = u_1 + u_2$.\\
\noindent
To go from $\mathcal{L}[u]$ to $\mathcal{L}^*[u]$, integrate $\mathcal{L}[u]$ by parts and collect the boundary terms, collect them and then force them to be zero.
\noindent
\begin{itemize}
  \item Eigenfunctions of the adjoint problem have the same eigenvalues as the original problem, $Ly = \l y \implies \ex w, L^*w = \l w$.
  \item Eigenfunctions corresponding to different eigenvalues are orthogonal (fiddle with $\l_j \ip{y_j}{w_k}$ to get $\l_k \ip{y_j}{w_k}$.)
\end{itemize}
\noindent
To solve the BVP, $Lu = f(x)$ subject to $Bu = g(x)$, Solve the eigenvalue problem, then solve the adjoint eigenvalue problem. Then assume $u = \sum_i c_iy_i(x)$ where $\l_kc_k \ip{y_k}{w_k} = \ip{f}{w_k}$\\
\noindent
If $\l_k = 0$, then we have two cases. If $\ip{f}{w_k} = 0$, then infinitely many sols by adding $y_0$, if $\ip{f}{w_k} \ne 0$ then no sols for $f$.\\
\noindent
A SL operator is self adjoint and is the following,
\begin{center}
  $\displaystyle{\mathcal{L}[u] = -\di{}{x}\left( p(x)\di u x \right) + q(x)u}$
\end{center}
\noindent
where $\mathcal{L}[u] = \l r(x)y(x)$, $r$ is weighting function. Further, $\mathcal{L} = \mathcal{L}^*$. We can convert from a BVP ($\mathcal{L} = a_2u'' + a_1u' + a_0u = f$) to SL form. Apply integrating factor then,
  ${\mu a_2 u'' + \mu a_1 u' = -pu'' - p'u'}$.
Then find, $p = e^{\int \frac{a_2}{a_1}\mathrm{d}x}$, $\mu = -\frac{1}{a_2}e^{\int \frac{a_2}{a_1}\mathrm{d}x}$, $q = -\frac{a_0}{a_2} u e^{\int \frac{a_2}{a_1}\mathrm{d}x}$ and $g(x) = -\frac{f}{a_2}e^{\int \frac{a_2}{a_1}\mathrm{d}x}$.\\
\noindent
\textbf{Orthogonality}, $\int_a^b y_k(x)y_j(x)r(x)\mathrm{d}x = 0$ for $j \ne k$. All the evals are real (fiddle with inner prods). If $a < x < b$ is finite, all $\l_k$s are discrete and countable.\\
\noindent
We can decompose any function by a complete set of $\{y_k\}$, that is, $h(x) = \sum c_iy_i(x)$ where $c_j = \frac{\int_a^b h(x)y_j(x)r(x) \mathrm{d}x}{\int_a^b y_j^2(x)r(x)\mathrm{d}x}$\\
\noindent
If we have $p(x) > 0$, $r(x) > 0$ and $q(x) \ge 0$ on $a \le x \le b$ and BCs have $\a_1\a_2 \le 0$ and $\a_3\a_4 \ge 0$. Then $\l_k \ge 0$ for $k =1,2, 3, \dots$. Proof, just consider $\ip{y_k}{Ly_k - \l_kry_k} = 0$, unfold and integrate by parts.\\
\noindent
The $k^{th}$ eigenfunction will have $k$ zeroes on $a < x< b$.\\
\noindent
Two SL problems, $\tilde \l_k > {\l_k}$, if $\tilde p(x) \ge p(x)$, $\tilde q(x) \ge q(x)$, $\tilde r(x) \le r(x)$ plus $(\tilde a, \tilde b) \sub (a, b)$\\
\noindent
\textbf{Greens Funcs}: We can consider $c_k$ and the def of $\ip{\cdot}{\cdot}$ then say,
\begin{center}
  $\displaystyle{y = \int_a^b \sum_k \frac{w_k(t)y_k(x)}{\l_k \ip{y_k}{w_k}}f(t)\mathrm{d}t } = \int_a^b g(x, t)f(t) \mathrm{d}t$
\end{center}
and we call $g(x, t)$ the greens function. We can find it by defining,
\begin{center}
  $\displaystyle{\mathcal{L}g = \begin{cases}
    \mathcal{L}g_- = 0 & a < x < \xi \\
    \mathcal{L}_+g = 0 & \xi < x < b
  \end{cases}}$
\end{center}
and then we need two more BCs, so we impose,
\begin{center}
  $\displaystyle{\int_{\xi_-}^{\xi_+} \mathcal{L}g\mathrm{d}x} = \int_{\xi_-}^{\xi_+} \d(x - \xi)\mathrm{d}x = 1$
\end{center}
Plug in $\mathcal{L}g = a_n g^{(n)} + a_{n-1}g^{(n-1)} + \dots$ and let $\left.a_ng^{(n-1)}\right|_{\xi_-}^{\xi^+} = 1$ and $\left. g^{(j)}\right|_{\xi_-}^{\xi^+} = 0$ for $0 < j < n-2$.\\
\textbf{Distributions:} $T$ is a distribution if it is linear and continuous, i.e.\\
(i) $\langle T,\alpha \phi_1+\beta \phi_2 \rangle = \alpha \langle T,\phi_1 \rangle + \beta \langle T,\phi_2 \rangle$, $\forall \alpha, \beta \in \mathbb{R}$ and $\forall \phi_1, \phi_2 \in \co$.\\
(ii) If $\phi_n(x)$ is a sequence of test functions s.t. $\phi_n(x) \rightarrow 0$ as $n \rightarrow \infty$ then $\langle T, \phi_n \rangle \rightarrow 0$ as a sequence of real numbers. Then $\lim_{n \rightarrow \infty} \langle T, \phi_n \rangle = \langle T, \lim_{n \rightarrow \infty} \phi_n \rangle$.\\
Equivalent continuity condition (for checking): $\forall L>0$, $\exists C>0$ and $N\geq 0$ s.t.\\
$|\langle T,\phi \rangle | \leq C \sum_{m\leq N} \max_{x \in \mathbb{R}} |\phi^{(m)}(x)|$, $\forall \phi$ s.t. supp $\phi \subset [-L,L]$.\\
\textbf{Translation property:} $\langle T(x+\alpha),\phi(x) \rangle = \langle T(x), \phi(x-\alpha) \rangle$, $\forall \phi \in \co$.
\textbf{Distributional derivative:} $\langle T^{\prime},\phi \rangle = -\langle T ,\phi^{\prime} \rangle$, $\forall \phi \in \co$.\\
\textbf{Convergence} of $T_j$ to $T$ as $j \rightarrow \infty$ means $\lim_{j \rightarrow \infty} \langle T_j,\phi \rangle = \langle T, \phi \rangle$, $\forall \phi \in \co$.\\
If $T(\alpha)$ is a family of distributions with continuous parameter $\alpha$ then $T(\alpha) \rightarrow T(\alpha_0)$ for $\alpha \rightarrow \alpha_0$ means $\lim_{\alpha \rightarrow \alpha_0} \langle T(\alpha), \phi \rangle = \langle T(\alpha_0), \phi \rangle$, $\forall \phi \in \co$.\hfill


\end{document}
