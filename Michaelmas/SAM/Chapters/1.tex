% !TEX root = ../notes.tex


\section{Introduction}
Study methods for solving inhomogenous BVP of the following form,
$$ Lu = f(x) \qquad a < x < b $$
where $L$ is a linear differential operator,
$$ Lu = a_n \din n u x + a_{n-1}\din {n-1} u x + \dots + a_1 \di u x + a_0 u  $$
where $f(x)$ is a forcing function and some boundary conditions imposed at $x = a$ and $b$. \\

\noindent
The theory for linear boundary value problem is much richer than the theory for initial value problems, and there are many applications of linear boundary value problems. Some examples are,
\begin{itemize}
  \item Shooting an arrow
  \item Tracking the melting of a block of ice
  \item Propogation of vibrations on a suspension bridge
\end{itemize}

\noindent
The sort of questions of interest when constructing solutions are,
\begin{itemize}
  \item Can we construct a solution for a given boundary value problem for an arbitrary function $f(x)$?
  \item If there is a solution, is there always a solution? (Existence) Is it unique?
  \item How do the choice of the boundary conditions affect the solutions?
  \item Can we construct solutions when $a_k = a_k(x)$?
\end{itemize}


\section{Eigenfunction Methods}
Consider a differential operator $L$. If there exists functions $y_i(x)$ such that,
$$ \begin{cases}
  Ly_i(x) = \l_i y_i(x) \\
  y_i(a) = y_i(b) = 0
\end{cases}. $$
If we can do this, $y_i(x)$ is an \textbf{eigenfunction} and $\l_i$ the associated \textbf{eigenvalue} of $L$.
\begin{idea}
  We are talking about linear differential operator, we can exploit this by constructing a solution that is a superposition (or sum) of the eigenfunctions ($y_i$) of $L$. That is,
  $$ y(x) = \sum_i c_iy_i(x) $$
  for some $c_i$.
\end{idea}

\subsection{Function Spaces}
There is a natural question on whether we can approximate $f$ as a sum of eigenfunctions of differential operator. Consider an infinite dimensional space of reasonably well-behaved functions on $[a, b]$. We can introduce a set of linearly independent basis functions $y_n$, such that, any reasonable $f$ in the space can be written as a sum of the basis functions,
$$ f(x) = \sum_{k=1}^\infty f_ky_k(x). $$
An example of this, are fourier series.

\begin{ndefi}[Inner Product]
  $$ \ip u v = \int_a^b u(x)\bar v(x)\mathrm{d}x $$
\end{ndefi}

\subsubsection{Weighting Functions}
Consider the following eigenvalue problem,
$$ Ly_i = \l_i \rho(x) y_i(x). $$
Here $\rho(x)$ is a weighting function, $\rho(x) \in \R$ and one signed on $[a, b]$. Further,
$$ \ip u v = \int_a^b \rho(x)u(x)\bar v(x)\mathrm{d}x $$

\subsection{Adjoint Operator}
\begin{ndefi}[]
  For an operator, $L$, with homogenous BC, then the \textbf{adjoint problem} ($L^*, BC^*$) is defined by,
  $$ \ip{Ly}{w} = \ip{y}{L^*w}. $$
  The $BC^*$ are just the boundary conditions needed such that the above equality holds.
\end{ndefi}

\begin{note}
  We use integration by parts to transfer derivatives from $y$ to $w$. Further, the choice of boundary conditions for $L^*$ will become clear.
\end{note}

\begin{eg}
  Let,
  $$ \begin{cases}
    Ly = \dii y x + \a \di y x + \b y & \quad x \in (a, b) \\
    y(a) = 0, \quad y_x(b) - y(b) = 0
  \end{cases} $$
  Then,
  \begin{align*}
    \ip {Ly}{w} &= \int_a^b (y_{xx} + \a y_x + \b y)w \mathrm{d}x\\
    &= \left[ wy_x - w_xy + \a wy \right]_{x = a}^{b} + \int_a^b (w_{xx} - \a w_x + \b w)y \mathrm{d}x
  \end{align*}
  From this, we can write,
  $$ \begin{cases}
    L^*w = w_{xx} - \a w_x + \b w\\
    [wy_x - w_xy + \a wy]_{x = a}^b = 0
  \end{cases}. $$
  We now need to exploit the boundary conditions from above to find the adjoint boundary conditions. We get,
  $$ 0 = [w(b) - w_x(b) + \a w(b)]y(b) - w(a)y_x(a) $$
  This must be true for any $y(b)$ and any $y_x(a)$. Hence we must let, $w(a) = 0$ and $w_x(b) = (1 + \a)w(b)$ as our adjoint boundary conditions.
\end{eg}

\begin{note}
  If in the above, $\a = 0$, then $L^*w = w_{xx} + \b w$, which is just the same as $L$. Hence it is self-adjoint. Furthermore, the boundary conditions become, $w(a) = 0$ and $w_x(b) - w(b) = 0$, which are the same as the original problem.
\end{note}

\begin{ndefi}[Self-adjoint]
  If $L = L^*$ and $BC = BC^*$, then the problem is called \textbf{self-adjoint}. Also if $L = L^*$ but $BC \ne BC^*$ then we call the operator self-adjoint.
\end{ndefi}

\noindent
Here are some facts, which turn out to be useful,
\begin{itemize}
  \item Eigenfunctions of adjoint problems have the same eigenvalues as the original problem,
  $$ Ly = \l y \implies \exists w, L^*w = \l w $$
  \item Eigenfunctions corresponding to distinct eigenvalues are orthogonal, that is, if $Ly_j = \l_j y_j$ and $Ly_k = \l_ky_k$ then $\ip {y_j}{w_k} = 0$.
  \begin{proof}
    The proof goes as follows,
    \begin{align*}
      \l_k \ip{y_j}{w_k} &= \ip{y_j}{L^*w_k} \\
      &= \ip{Ly_j}{w_k} = \l_j\ip{y_j}{w_k}.
    \end{align*}
    That is, we have,
    $$ (\l_j - \l_k)\ip{y}{w_k} = 0 $$
  \end{proof}
\end{itemize}

\subsection{Solution Process}
Consider the BVP,
$$ \begin{cases}
  Lu = f(x) \\
  BC_1(a) = 0, \quad BC_2(b) = 0
\end{cases} $$
What is the method?
\begin{enumerate}
  \item Solve the eigenvalue problem,
  $$ Ly = \l y \qquad BC_1(a) = 0 = BC_2(b), $$
  which will give $\{\l_j, y_j(x)\}$
  \item Solve the adjoint eigenvalue problem,
  $$ L^*w = \l w \qquad BC_1^*(a) = 0 = BC_2^*(b),$$
  which will give $\{\l_j, w_j(x)\}$
  \item Seek a solution to the boundary value problem of the form,
  $$ y(x) = \sum_{j=1}^\infty c_iy_i(x), $$
  where the coeffients $c_i$ are determined as follows:
  \begin{align*}
    Lu = f \implies \ip{f}{w_k} &= \ip{Ly}{w_k}\\
     &=\ip{y}{L^*w_k}\\
     &=\l_k\ip{y}{w_k} \\
     &=\l_k \ip{ \sum_i c_iy_i}{w_k}  \\
     &=\l_k c_k \ip{y_k}{w_k} \\
  \end{align*}
  and so we have, $c_k = \frac{\ip{f}{w_k}}{\l_k \ip{y_k}{w_k}}$
\end{enumerate}


\begin{eg}
  Consider,
  $$ \begin{cases}
    Ly = y'' + 4y = f(x)\\
    y(0) = y(1) = 0
  \end{cases} $$
  To solve,
  \begin{enumerate}
    \item Solve the eigenvalue problem,
    $$ Ly = y'' + 4y = \l y, \text{ with } y(0) = 0 = y(1) $$
    and this solves to, $y_n(x) = \sin (n\pi x)$, with $\l_n = n^2\pi^2 + 4$.
    \item Solve the adjoint problem,
    $$ L^*w = w'' + 4w = \l w, \text{ with } w(0) = 0 = w(1)$$
    and this solves to, $w_n(x) = \sin (n\pi x)$, with $\l_n = n^2\pi^2 + 4$.
    \item Seek a solution $y(x) = \sum_n c_ny_n(x)$ where, $c_n = \frac{\ip{f(x)}{w_n}}{\l_n\ip{y_n}{w_n}}$.
  \end{enumerate}
\end{eg}

\begin{exercise}
  Follow the same procedure for the BVP,
  $$ \begin{cases}
    Ly = x^2 y'' - 2xy' = f(x) \\
    y(1) = 0 = y(1)
  \end{cases} $$
\end{exercise}

\subsection{Solution Process for inhomogenous BC}
The problem in question is,
\begin{equation}
  \begin{cases}
    Lu = f(x) \\
    B_i u = \g_i
  \end{cases}\tag{$*$}\label{eqn:IBVP}
\end{equation}
There are two solution methods, one is decomposition and the other non-decomposition,
\begin{enumerate}
  \item Split the solution into two its, $u = u_1 + u_2$ where,
  $$ Lu_1 = f(x) \quad B_iu_1 = 0 $$
  $$ Lu_2 = 0 \quad B_iu_2 = \g_i $$
  Then by linearity $u = u_1 + u_2$ solves our problem \eqref{eqn:IBVP}
  \item We seek solutions of the form, $u = \sum c_ju_j$ where $\{\l_j, u_j\}$ are the eigensolutions of the linear operator with homogenous boundary conditions, i.e. $Lu_i = \l_i u_i$ and $B_1u_i = 0 = B_2u_i$. In this case, care is needed iwth BCs when using the inner product to determine the coefficients $c_j$.
\end{enumerate}

\begin{eg}
  $$ \begin{cases}
    y'' = f(x) & x \in (0, 1)\\
    y(0) = \a, y(1) = \b
  \end{cases}. $$
  With option 2,
  \begin{enumerate}
    \item Solve the eigenvalue problem,
    $$ Ly = y'' = \l y \quad y(0) = 0 = y(1), $$
    which has solution $y_k(x) = \sin (k\pi x)$, where $\l_k = -k^2\pi^2$.
    \item Solve the adjoint eigenvalue problem. The problem is self-adjoint,
    $$ w_k = y_k = \sin(k\pi x) $$
    \item Form an inner product of $y$ with $w_k$:
    \begin{align*}
      y'' = f(x) \implies \int_0^1 y''w_k\mathrm{d}x &= \int_0^1 f(x)w_k\mathrm{d}x \\
      (y'w_k - yw_k')_{x=0}^1 + \int_0^1 w_k''y \mathrm{d}x &= \int_0^1 w_kf \mathrm{d}x \\
      [y'w_k - yw_k']_{x=0}^1 + \l_k\int_0^1 yw_k \mathrm{d}x &= \int_0^1 w_kf\mathrm{d}x \\
      [y'w_k - yw_k']_{x=0}^1 + \l_kc_k\int_0^1 y_kw_k \mathrm{d}x &= \int_0^1 w_kf\mathrm{d}x \\
      [y'w_k - yw_k']_{x=0}^1 - k^2\pi^2c_k\int_0^1 \sin^2 (k\pi x) \mathrm{d}x &= \int_0^1 w_kf\mathrm{d}x \\
    \end{align*}
    Given $w_k(x) = \sin(k\pi x)$, we get,
    $$ [y'w_k - yw_k']_{x=0}^1 = -k\pi (-1)^k \b + k\pi \a $$
    and so,
    $$ c_k = -\frac{2}{k^2\pi^2} \int_0^1 \int_0^1 f(x)\sin (k\pi x) + \frac{2}{k\pi} (\a - (-1)^k \b) $$
  \end{enumerate}
\end{eg}
