% !TEX root = ../notes.tex

\subsection{Sobolev Spaces}

Consider,
$$ \begin{align*}
  -u''(x) = f(x)\\
  u(a) = A, \quad u(b) = B
\end{align*} $$
If we say that $u\in \mathcal{C}^2([a, b])$, then in the classical sense then this solution doesn't have a solution. However consider some $v$, that vanishes at the endpoints. Then,
$$ \int_a^b u'(x)v'(x) \mathrm{d}x = \int_a^b f(x)v(x)\mathrm{d}x $$
and then this is a new definition of a solution. This is a weak solution.\\

\noindent
Suppose that $u$ is locally integrable\footnote{This means that a function is integrable on every subset of $\o$.} on $\O$ for each bounded open set $\o$, with $\bar \o \sub \O$. Suppose also that there exists some $w_\a$ locally integrable on $\O$ such that,
$$ \int_\O w_\a(x)v(x) \mathrm{d}x = (-1)^{|\a|}\int_\O u(x)D^\a v(x) \quad \forall v \in C_0^\infty (\O). $$
Then $w_\a$ is called the weak derivative of $u$ (of order $|\a|$) and we write $w_\a = D^\a u(x)$.

\begin{eg}
  Let $\O = \R$ and let $u(x) = (1 - |x|)_+$ ($u(x) > 0$). Clearly $u$ isn't differentiable at $0, \pm 1$. However, $u$ is locally differentable and so will possibly have a weak derivative,
  \begin{align*}
    \int_{-\infty}^\infty uv' dx = \int_{-\infty}^\infty (1 - |x|)_+ v' dx &= \int_{-1}^1 (1 - |x|)v'(x) dx \\
    &= \int_{-1}^0 (1 + x)v' dx + \int_0^1 (1 - x)v'(x) dx \\
    &= \int_{-1}^0 (-1)v(x)dx + \int_0^1 v(x)dx \\
    &= - \infty_{-\infty}^\infty w(x)v(x) dx
  \end{align*}
  where,
  $$ w = \begin{cases}
    0, & x < -1 \\
    1, & x \in (-1, 0) \\
    -1, & x \in (0, 1) \\
    0, & x > 1
  \end{cases} $$
\end{eg}

\noindent
Let $k$ be a nonnegative integer. We define,
$$ H^k(\O) := \{ u \in L_2(\O) : D^\a u \in L_2(\O), |\a| \le k\} $$
$H^k(\O)$ is called a Sobolev space of order $k$; it is equipped with the Sobolev norm,
$$ \norm{u}_{H^k(\O)} = \left( \sum_{|a| \le k} \norm{D^\a u}^2_{L_2(\O)} \right)^{\frac{1}{2}} $$
and inner product,
$$ (u, v)_{H^k(\O)} := \sum_{|a| \le k} (D^\a u, D^\a v) $$

With this inner product, $H^k(\O)$ is a hilbert space. Letting,
$$ |u|_{H^k(\O)} := \left( \sum_{|a| = k} \norm{D^\a u}^2_{L_2(\O)} \right) $$
and we can write,
$$ \norm{u}_{H^K(\O)} = \left( \sum_{j=0}^k |u|^2_{H^j(\O)} \right)^{\frac{1}{2}} $$
and this is the Sobolev seminorm.

\noindent
We define a special sobolev space,
$$ H_0^1(\O) := \{u \in H^1(\O) : u = 0 \text{ on } \partial\O\}. $$
In Lesbegue you can change points on measure zero, the boundary is also measure zero. The trace theorems tell us what the implication of doing this happens.

\begin{nlemma}[Poincar\'e-Friedrichs Inequality]
  Suppose that $\O$ is a bounded open set in $\R^n$ and let $u \in H_0^1(\O)$, then, there exists a positive constant $c_*(\O)$ independent of $u$, such that,
  $$ \int_\O u^2(x) dx \le c_*\sum_{i=1}^n \int_\O |\partial_{x_i} u(x)|^2 $$
\end{nlemma}
\begin{proof}[]
  We shall the case for a square, that is, $\O = (a, b) \times (c, d)$. Then,
  $$ u(x, y) = u(a, y) + \int_a^x \partial_x u(\xi, y)d\xi = \int_a^x \partial_x u(\xi, y)d\xi \quad c < y < d $$
  Then from the Cauchy Schwartz, we get,
  $$ \int_\O |u(x, y)|^2 \le \frac{1}{2}(b - a)^2 \int_\O |\partial_x u(x, y)|^2 dx dy  $$
  and now we do it in the $y$ direction,
  $$ \int_\O |u(x, y)|^2 \le \frac{1}{2}(d - c)^2 \int_\O |\partial_y u(x, y)|^2 dx dy  $$
  Then we divide by the $(b - a)^2/2$ and $(d - c)^2/2$ and then just add them,
  $$ \int_\O u^2(x) dx \le c_* \int_\O (|\partial_x u| ^2 + \partial_y u|^2)dxdy $$
\end{proof}