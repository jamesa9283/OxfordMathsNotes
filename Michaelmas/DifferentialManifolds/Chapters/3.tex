% !TEX root = ../notes.tex

\subsection{Tangent Vectors and Tangent Space}
Let $X$ be a manifold and $x \in X$. We define a vector space $T_xX$ called the \textbf{tangent space} to $X$ at $x$. Elements $v \in T_xX$ are the \textbf{tangent vectors}. Heuristically they point in some direction. we think of them as some velocity of a point moving in $X$.
\begin{ndefi}[Tangent Vector]
  Let $X$ be a manifold and $x \in X$. A \textbf{tangent vector} at $x$ is a linear map $v : C^\infty(X) \to \R$ satisfying some Leibnitz rule, $v(ab) = a(x)v(b) + b(x)v(a)$ for all $a, b \in C^\infty(X)$.
\end{ndefi}
\noindent
We notice that this is a linear map and to we have a vector space of tangent vectors. This is a vector subspace of $C^\infty(X)^*$ (the vector space dual).

\begin{nprop}
   Let $X$ be an $n$-manifold $(U, \phi)$ be a chart on $X$, and $(u_1,\dots, u_n) \in U$ with $\phi(u_1,\dots, u_n) = x \in X$. Then $v : C^\infty(X) \to \R$ is a tangent vector if and only if it is od the form,
   $$ v(a) = \sum_{i=1}^n v_i \left.\pd{}{x_i} (a \circ \phi)\right|_{(u_1, \dots u_n)} $$
   for some unique $v_1, \dots, v_n \in \R$. Hence $T_xX \cong \R$, where $(x_1, \dots, x_n)$ are local coordinates of $X$ near $x$.
\end{nprop}
\begin{proof}
  For the `if' part, take $v_1, \dots, v_n \in \R$ and set, $v(a) = \sum_{i=1}^n v_i \left.\pd{}{x_i} (a \circ \phi)\right|_{(u_1, \dots u_n)}$ for $a \in C^\infty(X)$. Then $v(ab) = a(x)b(v) + v(a)b(x)$ follows from product rule of differentiation, so $v$ is a tangent vector. For the `only if' part, we can define some smooth $d_1, \dots, d_n : X \to \R$ with $d_i \circ \phi (x_1, \dots, x_n) = x_i - u_i$ in an open neighbourhood of $x$ in $X$. Let $v \in T_xX$, and set $v_i = v(d_i)$ for $i = 1, \dots, n$.
  Using Taylor's Theorem, for $a \circ \phi : U \to \R$ at $(u_1, \dots u_n)$ we can write,
  $$ a = a(x) \cdot 1 + \sum_{i=1}^n \left.\pd{}{x_i} (a \circ \phi)\right|_{(u_1, \dots, u_n) \cdot d_i} + \sum_{i,j = 1}^n F_{ij} \cdot d_i \cdot d_j + g, $$
  where $F_{ij} : X\to \R$ and $g : X \to \R$ are smooth with $g = 0$ in an neighbourhood of $X$. We write $g = g \cdot (1 - c)$ where $c = 1$ at $x \in X$. So,
  \begin{align*}
    v(a) &= a(x)v(1) + \sum_{i=1}^n \left.\pd{}{x_i}(a \circ \phi) \right|_{(u_1, \dots, u_n)} \cdot v_i + \sum_{i,j = 1}^n v((F_{ij}d_i)d_j) + v(g(1 - c)) \\
    &= \sum_{i=1}^n \left.\pd{}{x_i} (a \circ \phi) \right|_{(u_1, \dots, u_n)}v_i
  \end{align*}
\end{proof}

\begin{eg}
  Let $g : (-\e, \e) \to X$ be smooth with $\g(0) = x$. Define $v : C^\infty(X) \to \R$ by $v(a) = \dit (a \circ \g(t)) = 0$. Then using the product rule we see $v \in T_xX$. So the velocity of a moving point $\g(t) \in X$ is a tangent vector at $\g(0)$.
\end{eg}

\begin{ndefi}[Covariantly Functorial]
  Let $f : X \to Y$ be a smooth map of manifolds and $x \in X$ with $f(x) = y$. Define $T_xf : T_xX \to T_xY$ by $(T_xf)(a) : a \mapsto v(a \circ f)$, for $v \in T_xX$ and $a \in C^\infty(Y)$. This is well defined. If $g : Y \to Z$ is smooth with $g(y) = z$ then $T_x(g \circ f) = T_y g \circ T_xf : T_xX \to T_zZ$. So tangent spaces are covariantly functorial.
\end{ndefi}

\begin{remark}
   Let $X = Y = \R$, then the tangent spaces are also naturally indentified with $\R$ by the basis of $\partial_x$ and $\partial_y$. Hence it can be proved that, $T_x f = \di f x$.
\end{remark}

\subsection{Cotangent Spaces and 1-forms}
\begin{ndefi}[]
  Let $X$ be a manifold and $x \in X$. Then the \textbf{cotangent space} $T_x^*X$ to be the dual vector space $(T_xX)^*$.
\end{ndefi}
\noindent
Elements of $T_x^*X$ are called $1$-forms. If $(x_1, \dots, x_n)$ are local coordinates on $X$ near $x$ then $\pd{}{x_1}, \dots, \pd{}{x_n}$ are a basis for $T_xX$. We write $dx_1, \dots, dx_n$ for the dual basis for $T_x^*X$. If $f : X \to Y$ is smooth and $x \in X$
with $f(x) = y$, we write $T_x^*X : T_y^*Y \to T_x^*X$ for the linear map dual to $T_xf : T_xX \to T_XY$. For $g : Y \to Z$ smooth with $g(y) = z$ we have $T_x^* (g\circ f) : T_x^*X \circ T_y^*g$, so cotangent spaces are \textbf{contravariantly functorial}.

\begin{nprop}
   Let $X$ be a manifold and $x \in X$. We write $I_x = \{a \in C^\infty : a(x) = 0\}$, an ideal in $C^\infty(X)$. We write $I_x^2$ for the vector subspace of $C^\infty(X)$ generated by $ab$ for $a,b  \in I_x$, also an ieal in $C^\infty(X)$. Then there is a canonical isomorphism $T_x^*X \cong C^\infty(X) / (\gen 1 _R \oplus I_x^2)$. If $(x_1, \dots, x_n)$ are local coordinates of $X$ near $x$, them $\gen 1_R \oplus I_x^2$ is the kernel of some surjective linear map
   $C^\infty(X) \to \R$ mapping $a \mapsto \left(\left.\pd{a}{x_1}\right|_x, \dots, \left.\pd{a}{x_n}\right|_x\right)$
\end{nprop}
\begin{proof}
  By definition, $T_xX \subset C^\infty(X)^*$. Thus there is a natural isomorphism, $T_x^* \cong C^\infty(X)/W$ where $W \subset C^\infty(X)$ is the vector subspace of $a \in C^\infty(X)$ with $v(a) = 0$ for all $v \in T_xX$, and the dual pairing $T_x^*X \times T_x^*X \to \R$ maps $(a + W, b) \mapsto v(a)$. If $(x_1, \dots, x_n)$ are local coordinates at $x$, then $\pd{}{x_1}, \dots \pd{}{x_n}$ are a basis for $T_xX$.
  So $W$ is the kernel of $C^\infty(X) \to \R^n$ mapping the basis for $a \mapsto \left(\left.\pd{a}{x_1}\right|_x, \dots, \left.\pd{a}{x_n}\right|_x\right)$. Usig Taylors Theorem, we can also see that $W = \gen 1_R \oplus I_x^2$.
\end{proof}

\begin{ndefi}[Derivative]
  Let $X$ be a manifold, $x \in X$ and $a \in C^\infty (X)$. Define $d_xa \in T_x^*X$ to be the linear map $T_xX \to \R$ mapping $v \mapsto v(a)$. Equivalent under the isomorpihsm defined above, $d_xa $ is $a + (\gen 1_R \oplus I_x^2)$. We call $d_xa$ the \textbf{derivative} of $a$.
\end{ndefi}

\noindent
If $(x_1, \dots, x_n)$ are local coordinates on $X$ near $x$, and $dx_1, \dots, dx_n$ are the corresponding basis for $T_x^*X$, then $d_xa = \left.\pd{a}{x_1}\right|_x dx_1 + \dots + \left. \pd{a}{x_n}\right|_x dx_n$. But $dx_a$ makes sense without choosing coordinates.